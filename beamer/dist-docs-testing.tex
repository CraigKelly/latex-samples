\documentclass[t,handout]{beamer}

\usepackage[symbol*]{footmisc}
\DefineFNsymbolsTM{myfnsymbols}{% def. from footmisc.sty "bringhurst" symbols
  \textasteriskcentered *
}%
\setfnsymbol{myfnsymbols}

\usepackage{verbatim}
\usepackage{beamerthemeCopenhagen}
\beamertemplatenavigationsymbolsempty
\usepackage{amsmath}

\title{Dist, Docs, and Testing with OSS on GitHub}
\author{Craig Kelly}
\date{\today}

\begin{document}

\frame{\titlepage}

\section{Introduction}
    \subsection{Me and GLUDB}
        \frame {
            \frametitle{Hello}
            I'm Craig, a Research Developer for the Institute for Intelligent Systems
            at the University of Memphis (http://www.memphis.edu/iis/).

            \vspace{\baselineskip}

            Our mission depends on an interdisciplinary approach that brings together
            researchers from many different research areas in the cognitive sciences,
            including biology, communication sciences and disorders, computer science,
            education, engineering, linguistics, philosophy, physics, and psychology.
        }

        \frame {
            \frametitle{Why Does This Matter?}
            Grad students!

            \vspace{\baselineskip}

            They generally aren't experienced coders. In some cases they are just
            learning to program. And they're generally more concerned with DOE
            than making the database work. So...
        }

        \frame {
            \frametitle{What is GLUDB?}
            (And why should you care?)
            \begin{itemize}
                \item Simplified for the demands of working academics and grad students
                \item Annotate and go!
                \item Active Record (-ish)
                \item Documented-oriented data store with support for indexing
                \item Includes optional functionality for object change history and backups
                \item Allows easily moving from dev to server to cloud
                \item Supports sqlite, MongoDB, Google Cloud Datastore, and Amazon DynamoDB
            \end{itemize}
        }

        \frame {
            \frametitle{Is GLUDB for you?}
            Probably not.

            \vspace{\baselineskip}

            If you want a good, solid ORM then use SQLAlchemy. Or Django's ORM
            is you're Django'ing. Especially if you care about migrations.

            \vspace{\baselineskip}

            GLUDB can be useful for rapid iteration, hobby projects, or when
            you need a nice ``object store''
        }

    \subsection{Overview}
        \frame {
            \frametitle{Dist, Docs, and Testing}
            \begin{itemize}
                \item Dist via PiPy (pipy.python.org)
                \item Docs via Read The Docs (readthedocs.org)
                \item Testing via Travis CI
            \end{itemize}
        }

\section{Read The Docs}
    \subsection{Overview}
        \frame {
            \frametitle{Overview}
            TODO
        }
    \subsection{Implementation}
        \frame {
            \frametitle{How to Implement}
            TODO
        }
    \subsection{Example from GLUDB}
        \frame {
            \frametitle{Working Example}
            TODO
        }

\section{Travis CI}
    \subsection{Overview}
        \frame {
            \frametitle{Overview}
            TODO
        }
    \subsection{Implementation}
        \frame {
            \frametitle{How to Implement}
            TODO
        }
    \subsection{Example from GLUDB}
        \frame {
            \frametitle{Working Example}
            TODO
        }

\section{PyPI}
    \subsection{Overview}
        \frame {
            \frametitle{Overview}
            TODO
        }
    \subsection{Implementation}
        \frame {
            \frametitle{How to Implement}
            TODO
        }
    \subsection{Example from GLUDB}
        \frame {
            \frametitle{Working Example}
            TODO
        }

\end{document}

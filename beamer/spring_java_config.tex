\documentclass[t,handout]{beamer}

\usepackage[symbol*]{footmisc}
\DefineFNsymbolsTM{myfnsymbols}{% def. from footmisc.sty "bringhurst" symbols
  \textasteriskcentered *
}%
\setfnsymbol{myfnsymbols}

\usepackage{verbatim}
\usepackage{beamerthemeCopenhagen}
\beamertemplatenavigationsymbolsempty
\usepackage{amsmath}

\title{Spring Java Config}
\author{Craig Kelly}
\date{\today}

\begin{document}

\frame{\titlepage}

\section{Spring Java Config}

    \subsection{Introduction}
        \frame
        {
          \frametitle{XML Config for us today}
          \begin{itemize}
          \item Fairly standard XML file detailing the beans we need
          \item Generally fine, but any ``dynamic'' action is limited to
                factory beans and the like
          \end{itemize}
          {\tiny
            \verbatiminput{sjcxmlsample1.xml}
          }
        }

        \frame
        {
            \frametitle{Autowiring is Cool, too}
            \begin{itemize}
                \item Annotated classes are ``auto-wired'' on context startup
                \item Much of the configuration is inferred by scanning for annotated classes
            \end{itemize}
            \begin{block}{Want More Info?}
                See Raja's presentations on Spring Annotations!
            \end{block}
        }

        \frame
        {
          \frametitle{Java Config is an alternative}
          \begin{itemize}
          \item Replace the XML file with a Java class
          \item Get all the power of executing custom Java code at context startup
          \item Compatible with existing XML config and Autowiring
          \end{itemize}
          {\tiny
            \verbatiminput{sjcjavasample1.java}
          }
        }

    \subsection{Details}
        \frame
        {
            \frametitle{Basic Annotations}
            \begin{itemize}
                \item @Configuration is used to annotate a class and specify
                      that it can be used for configuring a context
                \item @Bean is used to annotate a method indicating that the
                      method creates a bean
            \end{itemize}
            \begin{alertblock}{Build Alert}
                Using these annotations means that you WILL need cglib
            \end{alertblock}
        }

        \frame
        {
            \frametitle{Some Restrictions}
            \begin{itemize}
                \item @Configuration classes must be non-final
                \item @Configuration classes must be non-local (may not be
                      declared within a method)
                \item @Configuration classes must have a default/no-arg
                      constructor and
                \item @Configuration classes may not use @Autowired
                      constructor parameters.
                \item Any nested configuration classes in a @Configuration
                      class must be static
            \end{itemize}
        }

        \frame
        {
            \frametitle{Loading the Context}
            If you can load a context with XML, then there is an alternative
            for Java Config.  Take the simplest example:
            {\tiny
                \verbatiminput{sjcstartxml.java}
            }

            There is an equivalent for Java Config.  Let's assume that the
            Java class that contains our context configuration is named
            JavaAppContext.
            {\tiny
                \verbatiminput{sjcstartjava.java}
            }
        }

        \frame
        {
            \frametitle{Spring Magic for Java Config, Part 1}
            Consider this example.  There are only two instances in the context:
            the catHat bean, and the someThing bean.  The catHat bean has two
            references to the \textbf{same} instance of someThing.
            {\tiny
                \verbatiminput{sjcxmlsample2.xml}
            }
        }

        \frame
        {
            \frametitle{Spring Magic for Java Config, Part 2}
            Here's the same example done with Java Config.  someThing appears
            to be called twice, so you might expect the catHat bean to have
            two different objects.  Luckily, the Spring annotation handles this
            so our Java code ``behaves'' like the XML context.
            \footnote[frame]{\tiny{Yes, I realize that this is also a radical
            reinterpretation of a classic piece of literature.}}

            {\tiny
                \verbatiminput{sjcjavasample2.java}
            }
        }

        \frame
        {
            \frametitle{Sample Project}

            A sample project is available demostrating all of the above.  There
            is even a unit test demonstrating that an XML context and a Java
            Config context produce the same results.  The sample project will
            be available in the same directory that you found this presentation.

            \begin{alertblock}{Build Alert}
                It's a Maven project!
            \end{alertblock}
        }

\end{document}

\documentclass[letterpaper]{article}

\usepackage{fullpage}
\usepackage{amsmath}

\title{The Business Case for a Knowledge Engineer}
\author{Craig Kelly}
\date{\today}

\begin{document}

% Force pdflatex to properly use letter as page size (instead
% of defaulting to A4)
\special{papersize=8.5in,11in}
\setlength{\pdfpageheight}{\paperheight}
\setlength{\pdfpagewidth}{\paperwidth}

%%%%%%%%%%%%%%%%%%%%%%%%%%%%%%%%%%%%%%%%%%%%%%%%%%%%%%%%%%%%%%%%%%%%%%%%%%%%
%%%%%%%%%%%%%%%%%%%%%%%%%%%%%%%%%%%%%%%%%%%%%%%%%%%%%%%%%%%%%%%%%%%%%%%%%%%%

\maketitle

\begin{abstract}
The Inventory Management department should be expanded to include a position
for knowledge creation and analytical development.  This position would allow
for increased strategic positioning, business growth, and informed decision
making.  In addition, it would provide much-needed support to the various
business functions in need of analytical services.

The position would include the responsibility to hand off deliverables
to I.T. for implementation, simultaneous development of multiple projects, and
responding to ad hoc queries about various business metrics.
As a result, the position requires a broad mix of skills including but
not limited to computational and mathematical skills, software engineering,
local business knowledge, the ability to cross departmental boundaries, and
the willingness to leverage the expertise of AMS and GE at large.
\end{abstract}


%%%%%%%%%%%%%%%%%%%%%%%%%%%%%%%%%%%%%%%%%%%%%%%%%%%%%%%%%%%%%%%%%%%%%%%%%%%%
%%%%%%%%%%%%%%%%%%%%%%%%%%%%%%%%%%%%%%%%%%%%%%%%%%%%%%%%%%%%%%%%%%%%%%%%%%%%

\section{The Case}

    \subsection{From Data to Knowledge}

    Our company culture has always included a deep respect for data.  However,
    as many have noticed, we are often data-rich but knowledge poor.  The
    excellent effort at every level in every function to record and maintain
    the transactional data generated on a daily basis often exhausts our
    resources.  As a result, very little time and effort is regularly expended
    to generate knowledge from this wealth of data.

    Our deep expertise in our industry, a detail-oriented IT infrastructure,
    and well-developed knowledge tool skills (such as Excel use) only
    exacerbate the problem.  Questions are often asked, answered, and
    discarded  before the knowledge generated can be captured.  In addition,
    those who lack access to the appropriate data and skills are often at a
    loss to access this wealth of knowledge.

    In addition, there are tremendous opportunities for new analytical
    projects to leverage our data assets.  These projects could:
    
    \begin{itemize}
        \item Drive efficiency by optimizing various business variables like
        pricing, product mix, and expected inventory impairment.

        \item Allow limited, but cost-free business experimentation.  Analytical
        models are difficult to build and verify, but allow reasonable answers
        to ad hoc questions from business leaders without changing current
        policies and procedures.

        \item Provide the knowledge necessary for strategic sales.  Analytical
        tools can be developed to help salespeople target new customers and drive
        growth with existing customers.  Non-obvious strategies might be uncovered,
        such as targeting specific equipment types in certain markets.
    \end{itemize}
    
    \subsection{From Ad hoc to IT}

    Many knowledge-based and analytical projects begin as a single ad hoc
    query or one-off question.  Often a decision maker simply wants to know
    how a particular metric is changing over time or affecting other variables
    and processes.  These ad hoc requests are important in and of themselves;
    unfortunately today they are often ignored or consume a disproportionate
    share of knowledge worker time.

    As a result, one goal for anyone dealing with knowledge creation or analytical
    modeling must somehow gather and curate the data, tools, and processes necessary
    for ad hoc queries.  In addition, these queries can frequently yield new
    metrics, new models, and new procedures for regularly answering the given
    question.  Eventually the new artifacts generated by the knowledge
    worker may become business critical or part of standard business functions.

    At this point, the work product should be handed off to Information Technology
    for ``proper'' system integration.  A knowledge worker in Inventory
    Management performing these roles and liaising with I.T. is an effective
    and efficient choice because:

    \begin{itemize}
        \item The specificity of the job results in sharpened skills and domain
        knowledge necessary to understand and execute complicated ad hoc knowledge
        requests.

        \item Ownership of both the question and the answer provide a single touch
        point for the resulting ``semi-automated'' process.

        \item Any processes handed off to I.T. have been kept in the ``language
        of IT'' from the beginning of the project, resulting in greatly decreased
        hand off and implementation times.
    \end{itemize}

    \subsection{Business/Market Intelligence}

    Most projects generated by a knowledge-based position could be classified
    as either internal business intelligence or external market intelligence.
    In the last thirty years, computer science, statistics, and related fields
    have made huge strides in various techniques for generating intelligence
    that can be leveraged for strategic decision making, tactical monitoring,
    and new product/market identification and evaluation.  Three general
    fields of inquiry that could be immediately leveraged are:

    \begin{itemize}
        \item Analysis, Modeling, and Simulation of various market conditions
        and processes including the quotation process, fleet dismantling,
        new product introduction, and macro market conditions.
        
        \item Forecasting, regression, and machine learning to predict the
        value of various business variables including customer inquiries,
        sales, repairs, and credit default.

        \item Clustering, comparisons, and link analysis to discover relationships
        between part types, aircraft types, vendor types, seasonality, market
        segments, and other business dimensions.
    \end{itemize}

    \subsection{Leverage}

    Like all GE businesses, AMS has access to a great deal of expertise
    available in the wider GE universe.  In addition, there is a huge pool
    of internal domain knowledge cultivated over years of experience.  Any role
    that includes knowledge management or analytical services must include
    a mandate to leverage both forms of expertise.  As a result:

    \begin{itemize}
        \item Looking across GE for best practices, data sources, and tools
        can provide some easy wins.

        \item Further examination of currently used external assets, including
        GECAS models and GRC deliverables, could provide extra value not currently
        being utilized.

        \item Seeking input across all AMS functions for large analytical projects
        would insure any output and deliverables are not necessarily lopsided.
        In addition, new ideas for analytical projects could be uncovered.
    \end{itemize}    

%%%%%%%%%%%%%%%%%%%%%%%%%%%%%%%%%%%%%%%%%%%%%%%%%%%%%%%%%%%%%%%%%%%%%%%%%%%%
%%%%%%%%%%%%%%%%%%%%%%%%%%%%%%%%%%%%%%%%%%%%%%%%%%%%%%%%%%%%%%%%%%%%%%%%%%%%

\section{Position Description and Requirements}

    \subsection{Business/Domain}

    Like any knowledge-based job (including I.T. and Finance), a great deal
    of internal business and domain knowledge would be required to effectively
    perform job duties.  However, a more important consideration would be the
    ability of a candidate to leverage the domain expertise in the
    Inventory Management department and across the company.

    \subsection{Computational and Mathematical Modeling and Analytics}

    A large part of the position will involve creating, evaluation, applying,
    and updating mathematical models that would form the basis of large
    analytics and projects.  In addition, application of computation modeling
    skills would be needed to complete most analytical projects involving
    ``modern'' simulation and machine learning techniques.
    
    \subsection{Software Engineering/Data Compliance}

    All project deliverables generated by this position must maintain
    compliance and be readily handed over to I.T. if they are deemed business
    critical. As a result, standard software engineering and project management
    skills should be utilized to maintain quality and reduce drag on hand-offs.

    \subsection{Possible Titles}

    \begin{itemize}
        \item Knowledge Engineer/Analyst
        \item Business Intelligence Engineer/Analyst
        \item Analytics Engineer/Analyst
        \item Financial Engineer/Analyst
        \item Computational Finance Engineer/Analyst
    \end{itemize}

%%%%%%%%%%%%%%%%%%%%%%%%%%%%%%%%%%%%%%%%%%%%%%%%%%%%%%%%%%%%%%%%%%%%%%%%%%%%
%%%%%%%%%%%%%%%%%%%%%%%%%%%%%%%%%%%%%%%%%%%%%%%%%%%%%%%%%%%%%%%%%%%%%%%%%%%%

\section{Possible Projects}

    This is a brief list of possible projects for the proposed position.

    \begin{description}

    \item[Ad Hoc Analytical Queries]
        Develop system to allow analytical exploration and deep dives without
        impacting ICS transactional processing or ``stealing'' I.T. time.

    \item[Inventory Query (AKA ``Project Amazon'')]
        Allow free exploration of current and pending inventory by a variety of
        criteria (like the shopping experience on amazon.com)

    \item[Dashboarding]
        Continuous metric identification refinement with little manual labor

    \item[``Gold Standard'' Data]
        Data cleaned, processed, and vetted for consistency, correctness, and
        compliance to be provided for both internal and external challenges.  For
        instance, forecasting data that AMS could use to test any forecasting
        ideas they have.  Similar, anonymized data could be provided to various
        crowdsourcing/consulting parties.

    \item[Pricing]
        Develop pricing prediction models for audits to see if a price was entered
        in error (NOT the same as automated pricing)

    \item[Portfolio Optimization]
        Using parameterized inputs (from expert what-if's, forecasting, inventory,
        skyline, etc.) determine product mix and its effect on forecasted sales
        versus an optimized product mix.  What are the deltas and what are the
        smallest changes to current planning with largest effect?

    \item[``Lever'' Modeling]
        Combination of models and simulation to allow exploration of ``levers
        we can pull'', including price elasticity, inventory levels, marketing
        (including controlled inventory listing on ILS).

    \item[``Micro'' Forecasting]
        Apply trending (especially for the bid forecast), re-model the forecast by
        making quotations explicit.

    \item[``Macro'' Forecasting]
        Find market saturation points for aircraft types, explore relationships to
        major indices (including fuel spot pricing), leverage GRC fleet/price
        curve, find ways to leverage the ``lag effect'' (aviation pax/cargo often
        lags certain economic indicators and we often lag aviation pax/cargo).

    \item[ILS Data Leverage]
        New models for using, interpreting, and driving various ILS statistics.

    \item[GRC Credit Modeling]
        Find way to apply EBER modeling with more timely financials. (Make the
        credit modeling GRC project pay off).

    \item[IDD News Story sources]
        Group similar news stories via language analysis for batch reading to
        lean out the current IDD process.

    \item[ACM Knowledge]
        Leverage the work done with IPC's and ``continuous trees'' for our 
        aircraft models.

    \item[Company Clustering]
        Find ways to identify new sales leads by locating companies similar to
        top performers today.

    \item[Upselling]
        Are certain parts more likely to sell when other parts are sold?  Can
        this knowledge be used for upselling on the fly?  Can alternate part
        information be leveraged?

    \item[Sales force game theory]
        Use statistical scoring techniques to determine an ordering of
        salespeople who would be best when dealing with certain product types,
        aircraft types, and/or regions.

    \item[Note Analysis]
        We create a tremendous amount of information via ICS ``notes''.  Can
        computational linguistics and natural language analysis be used to
        extract and formalize some of this information?

    \item[Regulatory Textual Analysis]
        Automated checking for possible problems (or market advantages)
        unique to our business created by the release of new FAA/EASA AD's
        or SAIB's.

    \end{description}

\end{document}

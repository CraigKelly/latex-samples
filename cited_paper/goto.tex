\documentclass[letterpaper,10pt,twocolumn]{article}

\usepackage{fullpage}
%\usepackage{cite}
\usepackage{url} %used by our custom annotated bibliography
\usepackage{amsmath, amsthm}

\usepackage{nameref} %to get nameref
\newcommand{\fullref}[1]{section \ref{#1} (\emph{\nameref{#1}})}

\usepackage[backend=bibtex,style=numeric,citestyle=numeric]{biblatex}
\bibliography{goto}

\begin{document}

% Force pdflatex to properly use letter as page size (instead
% of defaulting to A4)
\special{papersize=8.5in,11in}
\setlength{\pdfpageheight}{\paperheight}
\setlength{\pdfpagewidth}{\paperwidth}

%%%%%%%%%%%%%%%%%%%%%%%%%%%%%%%%%%%%%%%%%%%%%%%%%%%%%%%%%%%%%%%%%%%%%%%%%%%%%
% Title
%%%%%%%%%%%%%%%%%%%%%%%%%%%%%%%%%%%%%%%%%%%%%%%%%%%%%%%%%%%%%%%%%%%%%%%%%%%%%
\title{Goto Hell: the Need for Better Control Structures and Compilers}
\author{Craig Kelly}
\maketitle


%%%%%%%%%%%%%%%%%%%%%%%%%%%%%%%%%%%%%%%%%%%%%%%%%%%%%%%%%%%%%%%%%%%%%%%%%%%%%
% Introduction
%%%%%%%%%%%%%%%%%%%%%%%%%%%%%%%%%%%%%%%%%%%%%%%%%%%%%%%%%%%%%%%%%%%%%%%%%%%%%
\section{Introduction}

The unconditional branch instruction (called the ``goto'' statement) has been
a source of controversy for more than 30 years.
The necessity of the statement for efficiency and its harm to programs that
must be maintained by more than one person have been discussed at length.
Although the most famous pro and con opinions were written by Dijkstra and Knuth
respectively \cite{dijkstra,knuth}, many more articles, papers, and discussion
have considered the issue at length.
Many have considered the argument settled for quite some time, but there is
still some dissent.  In 1987, more than 20 years after the arguments began,
F. Rubin claimed that the belief that goto should be eliminated lacked
an adequate rational foundation \cite{rubin}.
That letter sparked yet another round of lively debates
\cite{jonsson} with no universally recognized winner.

Although it is interesting to review the particulars of such a debate,
the issue of the goto statement still holds relevance for the computing
profession today.
\emph{Ongoing issues with goto should be viewed as a need to improve control
 structures and compiler optimization techniques.}

\section{Justification}

Before continuing the story of goto, it is instructive to consider the relative
importance of the goto statement and in what scope it will be examined.

There is little doubt that the issue of goto is important to the computing community.
Some of the most influential scholars in the computing field have published
papers and articles concerned with nothing but the evaluation of a single small
instruction common to most procedural languages.
When Dijkstra published his now famous ``Go to statement considered harmful''
\cite{dijkstra} it was meant to be a short article in a journal.  However, Knuth
reports that it was published as a letter to the editor in an effort to speed
its publication \cite{knuth} (ostensibly to skip the peer-review process).

As Leavenworth put it,
the concept of the goto statement "does not appear in most formal systems of
computability theory, but does appear in programming language extensions of those
systems" \cite{leavenworth}.  One might argue then that goto appears in programming
languages because it is a conceit that has existed in some form for most of the
history of the theoretical computer; the Turing machine itself has the concept of
a goto statement.
\label{goto_von_neumann} However, Rojas has shown that there are only four
basic instructions needed for a von Neumann computer: load, store, increment,
and goto \cite{rojas}.  Since most modern computer systems are based on the
von Neumann architecture, the goto statement is here to stay in some form (even
if only as a microcode at the heart of a CPU's hardware).

\section{Scope of Inquiry}

Insomuch as programming languages can be viewed on a continuum from low to high
level, the scope of any consideration of goto much be constrained to a subset
of that spectrum.  At the very lowest level is hardware machine code and assembly
language.  As mentioned above, goto is intrinsic to von Neumann computers and so
much be viewed as necessary to the hardware instructions.  As a result, it is
not reasonable to consider the fate of the goto until the language under
consideration is at some minimum higher level.

At the other end of the spectrum are languages that are far enough removed
from procedural operation that even the consideration of a branching instruction
makes little sense.  For instance, most declarative languages (such as SQL) and
functional languages (such as Erlang) don't contain a concept related to a
procedural branch and so are outside any scope of consideration.
\footnote{Of course, there are declarative and functional languages implemented
via a procedural language.  While these procedural programs may be considered
part of the scope under consideration, their output and runtime concerns are
not. }



%%%%%%%%%%%%%%%%%%%%%%%%%%%%%%%%%%%%%%%%%%%%%%%%%%%%%%%%%%%%%%%%%%%%%%%%%%%%%
% overview
%%%%%%%%%%%%%%%%%%%%%%%%%%%%%%%%%%%%%%%%%%%%%%%%%%%%%%%%%%%%%%%%%%%%%%%%%%%%%
\section{Overview of Debate}

As a matter of history, Knuth claims \cite{knuth} that Peter Naur was the
first to publish about ``harmful go to's'':

\begin{quote}
If you look carefully you will find that surprisingly often a go to statement
which looks back really is a concealed for statement.  And you will be pleased
to find how the clarity of the algorithm improves when you insert the for clause
where it belongs...
\cite{naur}
\end{quote}

The most influential publications in the historical goto debate are Dijkstra's
``Go to statement considered harmful'' \cite{dijkstra} and Knuth's ``Structured
Programming with go to Statements'' \cite{knuth}. \footnote{In an odd twist
of fate, Knuth's paper is often better known for it's use of the phrase
``premature optimization is the root of all evil''.  Although often quoted,
most don't realize it comes from a paper defending goto on the grounds of
optimization!}  These two positions and the commentary surrounding them sum up
much of the debate surrounding the goto statement.

Dijkstra's claimed that ``[t]he goto statement as it stands is just too primitive;
it is too much an invitation to make a mess of ones program'' \cite{dijkstra}.
However, much of the article is dedicated to the conceptual handicap imposed
by a program containing goto's:

\begin{quote}
...our intellectual powers are rather geared to master static relations and that
our powers to visualize processes evolving in time are relatively poorly developed.
For that reason we should do (as wise programmers aware of our limitations) our
utmost to shorten the conceptual gap between the static program and the dynamic
process, to make the correspondence between the program (spread out in text space)
and the process (spread out in time) as trivial as possible.
\cite{dijkstra}
\end{quote}

Dijkstra goes on to develop a simple way of keeping enough data to restart a
program at any given point.  Leavenworth pointed out that Dijkstra's argument
can be considered in a stronger light; unrestricted goto's increase the
difficulty of the halting problem.  If goto's are removed, ``there are only
two ways a program may fail: either by infinite recursion, or by the repetition
clause.'' \cite{leavenworth}

Knuth's argument is often seen as a counterpoint to Dijkstra, and a pro goto
stance.  Although he does argue for the use of goto in some situations, he
also argues "for the elimination of go to's in certain cases" \cite{knuth}.
Additionally, Knuth quotes personal correspondence from Dijkstra \cite{knuth} that
reveals Dijkstra's fairly measured stance on goto usage:

\begin{quote}
Please don't fall into the trap of believing that I am terribly dogmatical about
[the go to statement].  I have the uncomfortable feeling that
others are making a religion out of it, as if the conceptual problems of
programming could be solved by a single trick, by a simple form of coding
discipline!  \cite{dijkstra-personal}
\end{quote}

Knuth's main arguments for using goto's are a series of patterns and examples
given a number of situations in which goto usage is clearer, more efficient, or
more desirable.  Leavenworth amalgamates these reasons and situations into three
justifications for the use of goto: synthesis of missing control structures,
efficiency, and abnormal exits \cite{leavenworth}.

The main example given for the case of missing control structures is the
simulation of a $case$ statement in a language (like PL/I) that has no such
structure but does have goto.  This particular argument may be given less weight
because goto removal and adding a case statement are both languages changes.
Adding a language feature provides backward compatibility, while removing a
feature from a programming language breaks existing programs.
Thus, any argument for goto involving the synthesis of a missing
control structure in a specific language must be viewed as a weaker version of
an argument to add that specific control structure.

Abnormal exits and efficiency are two concerns that are often related.  The
use of a goto in the case of an abnormal exit from a loop or subprogram is
often replaced with a structure involving the evaluation of a boolean variable.
On many platforms, the goto is a single machine instruction, while the boolean
test is often reduced to multiple instructions.


%%%%%%%%%%%%%%%%%%%%%%%%%%%%%%%%%%%%%%%%%%%%%%%%%%%%%%%%%%%%%%%%%%%%%%%%%%%%%
% complete removal
%%%%%%%%%%%%%%%%%%%%%%%%%%%%%%%%%%%%%%%%%%%%%%%%%%%%%%%%%%%%%%%%%%%%%%%%%%%%%
\section{Goto Removal}

As mentioned in \fullref{goto_von_neumann}, the goto construct is necessary at a
fundamental level in the von Neumann architecture; however, at a sufficiently
abstract level it becomes unnecessary (such as in functional languages).
Thus it is only natural to ask if we can do away with the goto entirely in
procedural programming languages.  Dijkstra himself pointed this
out \cite{dijkstra}.  Only two years earlier, Jacopini had shown that goto's
could be removed via a technique involving the transformation of flow
diagrams \cite{jacopini}.  However, Dijkstra also pointed out that the resulting
program was actually less clear than the original and so was actually worse
than a program with a goto.

Knuth covered the same issue with Jacopini's work.  However, work continued on
automated goto removal.  As Knuth also points out, Ashcroft and others created
a technique that in many cases provided a more readable goto-less transformation
\cite{ashcroft}. Unfortunately the technique may cause an exponential code growth.
This particular path of inquiry continued with Kosaraju's work on replacing
goto with features that do not have the same drawbacks (usually known by
the keywords break, continue, exit, etc) \cite{kosaraju}.
The work was especially important in that it proved that goto's
could be replaced by the specified commands with no extra computation provided
that the structures allow exiting an arbitrary number of nested loops.

Kosaraju's work may appear to apply to only those languages that have ``exit
control'' structures that allow escape from multiple nested loops.
However, this is achieved in most procedural languages by exiting the
current function, implying that function
refactoring can be using as part of the transformation technique.
Unfortunately, this removes the original guarantee of no increased computation.
The guarantee may be returned given a compiler that could compile the function called
as if it were a labeled block and not a subprogram.  Reliance on this compiler
feature (known as ``inlining'') is seen a great deal in modern $C$ programs,
which have a conspicuously low number of goto statements.  It could be argued
that this is a goto-removal technique that has gone undocumented but is fully
practiced by a the programming community.

Perhaps the largest boost to the goto-replacement argument came three years later
from McCabe's landmark paper describing cyclomatic complexity \cite{mccabe}.
Although cyclomatic complexity is a helpful software metric, the important
contribution of the paper from the ``goto perspective'' is the attempt to give
a formal definition of a non-structured program.  McCabe notes that Kosaraju
\cite{kosaraju} provided a proof concerning which flow graphs are reducible to
structured programs.  McCabe also points out that while this is a nice result,
practitioners need a way to formally identify a non-structured program.
He demonstrates that ``a structured program can be written by not `branching out of
or into a loop, or out of or into a decision'.'' \cite{mccabe}

This result provides the other side of goto replacement; having a formally
unstructured program is the cost of not removing goto's that violate loop
and decision boundaries.  Of course, McCabe also points out that ``there a few
very specific conditions when an unstructured construct works best'' \cite{mccabe},
pointing again to the necessity of efficiency over program correctness.
Wulf continues this questioning in a published argument against goto \cite{wulf}.
As he explains,
``...it is possible to eliminate the goto'' but it must be determined
``...whether it is practical to remove the goto'' \cite{wulf}.

However, Wulf demonstrates that it can be argued that goto's may have already
been removed without anyone noticing. That, in fact, if ``any rational
set of restrictions'' is placed on goto, it is ``equivalent to eliminating the [goto]
construct if an adequate set of other control primitives is provided'' \cite{wulf}.
Given that none of the presented efficiency arguments (from Knuth
or anyone else) involves drastic uses of goto that preclude ``rational''
restrictions, in might be argued that goto has already been conceptually
removed from any language with sufficient control primitives.

\label{wulf-bliss-compiler}
Wulf continues his argument by relating a programming language named Bliss that
he co-designed and used as a system language.  In addition to user applications,
Bliss was used to write an operating system and multiple compilers.  This situation
appears to be a multi-year experiment in the expunction of goto.  The
reported Bliss experience would appear to support this argument.  However, some
may argue that the systems could have been more efficient with some judicious
goto usage.  This argument ignores the fact that goto's were
used in the Bliss system, but were used exclusively by the compiler.
Wulf acknowledges the importance of a well-written compiler as a necessary
component in goto-less programming.  He states that ``the appropriate
mechanism for achieving this efficiency is a highly optimizing compiler, not
incomprehensible source code'' \cite{wulf}.
\footnote{Wulf also wrote that ``[m]ore computing sins are committed in the
name of efficiency (without necessarily achieving it) than for any other single
reason -- including blind stupidity.'' \cite{wulf}  This is not as relevant,
but it is certainly more amusing.}


%%%%%%%%%%%%%%%%%%%%%%%%%%%%%%%%%%%%%%%%%%%%%%%%%%%%%%%%%%%%%%%%%%%%%%%%%%%%%
% Better Control structures and Compilers
%%%%%%%%%%%%%%%%%%%%%%%%%%%%%%%%%%%%%%%%%%%%%%%%%%%%%%%%%%%%%%%%%%%%%%%%%%%%%
\section{Improved Control Structures and Compiler Optimization Techniques}

Wulf's reasoning makes explicit what has been implicit in most of the discussion
of the goto statement.  The attempt to remove goto from use can also be
interpreted as a request for better control structures and better compilers.
As often happens, this position is a return to the origins of the debate.  Knuth
himself pointed out that ``new types of syntax are being developed that provide
good substitutes for these... go to's...'' \cite{knuth}.  However, Knuth set
aside the possibility of an optimizing compiler handling the efficiency question
because it ``would have to be so complicated (much more so than anything we
have now)'' \cite{knuth}.  Although his proposed alternative (``Program
Manipulation Systems'') never came to fruition, optimizing compilers have
come quite a long way.

There have been a few notable attempts at better control structures.  One very
large attempt was described by Jonsson as a replacement for code patterns used
to replace goto statements (which he called ``goto-patches'') \cite{jonsson}.
This group of structures (known as ``pancode'' \cite{jonsson_pancode}), includes
several constructs that target situations usually thought to require a goto for
maximum efficiency. These include an $also$ statement for an explicitly short-circuited
logical conjunction and a $repeat$ statement which loops back to the nearest
$if$ statement.

Other constructs have also been proposed.  Wulf's Bliss language included a
``$leave \langle label \rangle$'' construct where the label specifies the name
of the program section to be left (or exited) \cite{wulf-leave}.  Conceptually
this construct has been implemented in languages as a labeled break statement;
perhaps the most popular language with this feature is Java.

Another construct that has found eventual use is Bochman's statements for
multiple loop exits \cite{bochman}.  This structure is an unlabeled break
(envisioned by Bochman as the keyword ``exitloop'') with special labels for
post-loop code execution.  There is an ``ended'' label for code to be executed
when the loop terminates normally, and an ``exited'' label for code to be
executed when the loop is terminated by a break statement.  This concept has
been implemented as an optional else clause for loops in Python.

Circumstantial evidence seems to imply that these constructs have indeed been
helpful in removing goto.  Neither Java nor Python provide a goto statement
\footnote{As a matter of interest, $goto$ is a reserved word in Java
\cite{java_keywords}.  However, Java does not implement goto,
which formally forbids the use of goto for any reason in Java.}.
Of course, a cynic may respond that these languages still have
efficiency problems since Python is interpreted and Java executes on a virtual
machine.  As a result, compiler technology must also be considered.

Certainly, Wulf's Bliss compiler mentioned in \fullref{wulf-bliss-compiler} is
an answer to the complaint.  However, relatively recent work (in 1994) by Erosa
and Hendren \cite{erosa} shows that optimizing programs after automatically
eliminating goto's is not only possible but a requirement for some compiler
optimizations.  The process described in the paper is an automated removal
of goto statements as described multiple times above.  The elimination of goto's
allows for multiple optimizations that would otherwise be impossible.  In most
situations (with one notable exception), performance was equivalent.

Other compiling techniques that require the elimination of goto's have been
described as well.  Allen et al described a process for eliminating
goto's for their vectorizing compiler project \cite{allen}.  Ammarguellat
described a process based on simultaneous equations to remove goto's for a
parallelizing compiler \cite{ammarguellat}.  These two efforts demonstrate
the need to remove goto's for optimal vectorization or parallelization.

These efforts are complementary because vectorization allows for parallel
instructions within a single flow of control, while parallelization allows
multiple flows of control to proceed simultaneously.  Gains from these types of
enhancements could far outweigh any performance penalties.  Consider a program
which contains a goto and can be executed in time $T$.  Further imagine that it
could be fully parallelized by one of the above techniques.  Assuming that
the compiler increases the execution time by a factor of $\alpha$ when the goto
removal operation is applied, the final execution time would be
$\alpha T \frac{1}{N} = \frac{\alpha T}{N}$ where $N$ is the degree of
parallelization (perhaps the number of processors).  As long as $N > \alpha$,
it is more efficient to remove goto's (in this idealized case).  As an example,
assume that $N = 4$ based on the current multicore chips available on the market
for home computers.  Unless the goto removal process quadruples the program run
time, it will always be more efficient to remove the goto statements.


%%%%%%%%%%%%%%%%%%%%%%%%%%%%%%%%%%%%%%%%%%%%%%%%%%%%%%%%%%%%%%%%%%%%%%%%%%%%%
% conclusion
%%%%%%%%%%%%%%%%%%%%%%%%%%%%%%%%%%%%%%%%%%%%%%%%%%%%%%%%%%%%%%%%%%%%%%%%%%%%%
\section{Conclusion}

In some sense, goto's must always be with us if only in machine code.  Of
course, this implies that at the very least someone must write a compiler
that emits goto instructions.  This special case aside, the general agreement
has always been that goto instructions are not good for most human-readable
programs.  Those arguing for the use of goto do so almost universally on
the basis of efficiency.

The proper response to this state of affairs would obviously be to create
alternatives to goto that are just as efficient.  Thus, any situation that
appears to require a goto should be considered a requirement for a control
structure that can be used instead.  This creates an ongoing requirement
to improve compiler optimizations so that these structures are at least as
efficient as goto statements and maximally as efficient as possible.

Ongoing work in this area has demonstrated that some forms of optimization
and parallelization require removal of goto's.  As a result, the combination
of automated parallelization and ubiquitous multicore hardware will mean
that goto removal is required for optimal program execution.   As a result,
the humble goto might become as rare and esoteric as interrupt suspension --
used only by programmers delving into assembly language to deal directly with the
physical CPU.


%%%%%%%%%%%%%%%%%%%%%%%%%%%%%%%%%%%%%%%%%%%%%%%%%%%%%%%%%%%%%%%%%%%%%%%%%%%%%
% Bibliography
%%%%%%%%%%%%%%%%%%%%%%%%%%%%%%%%%%%%%%%%%%%%%%%%%%%%%%%%%%%%%%%%%%%%%%%%%%%%%

\nocite{*} %ensure entire bib is shown
\printbibliography
\end{document}

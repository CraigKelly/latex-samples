\documentclass[letterpaper,10pt]{article}

\usepackage{fullpage}

\usepackage{amsmath}
\usepackage{amssymb}
\usepackage{amsthm}
\usepackage{braket}

%We would like cardinality of a set to require one (not four) commands
\newcommand{\card}[1]{\left\vert{#1}\right\vert}

\title{Set Notation Help}
\author{Craig Kelly}

\begin{document}

% Force pdflatex to properly use letter as page size (instead
% of defaulting to A4)
\special{papersize=8.5in,11in}
\setlength{\pdfpageheight}{\paperheight}
\setlength{\pdfpagewidth}{\paperwidth}

\setlength{\parindent}{0pt}
\setlength{\parskip}{6pt}

\maketitle

%%%%%%%%%%%%%%%%%%%%%%%%%%%%%%%%%%%%%%%%%%%%%%%%%%%%%%%%%%%%%%%%%%%%%%%%%%%%
%%%%%%%%%%%%%%%%%%%%%%%%%%%%%%%%%%%%%%%%%%%%%%%%%%%%%%%%%%%%%%%%%%%%%%%%%%%%

\section*{Some set notation}

Let $ Z $ be the set of integer $ \geq 0 $.
Show $ Z \times Z \times Z $ is countable by constructing the bijection
$ f:  Z \times Z \times Z \rightarrow \mathbb{N} $

I am assuming that $0 \notin \mathbb{N}$, 
so $\mathbb{N} = (1,2,3, \cdots)$ 


\begin{proof}

First we show that there is a bijection 
$ g: Z \times Z \rightarrow \mathbb{N} $.

blah, blah, blah

$ g: Z \times Z \rightarrow \mathbb{N} $
is a bijection and therefore $Z \times Z$ is countable.

We may now construct $ f:  Z \times Z \times Z \rightarrow \mathbb{N} $
via composition:

$ f(Z_{1}, Z_{2}, Z_{3}) = h(g(Z_{1}, Z_{2}), Z_{3}) $

\end{proof}


\begin{proof}

$ L1 = \set{M | \text{M accepts w if w contains the substring 10} } $
is undecidable

Blah, blah, blah

If M(w) accepts, then L(HWAMw) = $\set{w | \text{w contains the substring 10} }$

If M(w) rejects or fails to halt, then L(HWAMw) = $\emptyset$

Thus, we know
$(M,w) \in A_{TM} \Leftrightarrow HWAMw \in L1$

\end{proof}

\pagebreak


\begin{proof}
$ L2 = \set{M | \text{M accepts an odd number of strings} } $ is undecidable

$P(M)$ if $\card{L(M)}$ is odd.

$x \in \set{000, 111, 101}$

Blah, blah, blah

If M(w) accepts, then L(HWBMw) = $\set{000, 111, 101}$.  Note that 
$\card{L(HWBMw)} = 3$, which is odd.

If M(w) rejects or fails to halt, then L(HWBMw) = $\emptyset$, so
$\card{L(HWBMw)} = 0$

Thus, we know
$(M,w) \in A_{TM} \Leftrightarrow HWBMw \in L2$

\end{proof}


\end{document}

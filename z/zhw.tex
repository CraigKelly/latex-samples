\documentclass[letterpaper,11pt]{article}
\usepackage{ltcadiz}

%%side margin is 1.0in + oddside, so we're creating a 1 inch margin
\oddsidemargin 0.0in 
\topmargin 0.0in
\textwidth 6.5in

\usepackage{fullpage}

\title{Z Example Taken from Some Homework}

\begin{document}

\maketitle

\section*{Introduction}

The Class Manager's Assistant module must provide an interface for dealing
with students enrolling in a class.  Students enroll in a class.  The result
of the enrollment is specified by \emph{EnrollResponse}.  At some point in
time, a student may be tested; it is assumed that a student will pass the
test.  When a student is tested, the Class Manager's Assistant must provide
an appropriate \emph{TestResponse}.

\vspace{1 em}

Given Set [Student]

    $| size: \nat$
    \begin{zed}
    EnrollResponse :: = success | alreadyenrolled \\
    TestResponse ::= success | alreadytested | notenrolled \\
    \end{zed}

\section*{Abstract State Schema}

For a \emph{Class} in the Class Manager's Assistant, a student must enroll in
a class to be considered enrolled.  Once a student is enrolled, the student
may be tested.

    \begin{schema}{Class}
    enrolled: \bbold{P} Student \\
    tested: \bbold{P} Student
    \where
    tested \subseteq enrolled \\
    \# enrolled \leq size
    \end{schema}

\section*{Initialization of Abstract State}

It is initially assumed that a class is empty, and therefore has no students
enrolled.

    \begin{zed}
    [Class' | enrolled' = 0 ]
    \end{zed}

\section*{Operation - Enroll}

When a student requests to be enrolled, that student is added to the Class.

    \begin{schema}{EnrollOK}
    \Delta Class \\
    S? : Student \\
    R! : EnrollResponse \\
    \where
    S? \notin enrolled \\
    enrolled' = enrolled \union \left\{ S? \right\} \\
    R! = success \\
    \end{schema}

If a student has already enrolled, then that student can not enroll in the
Class again.  Please note that this would be regardless of the student's
test status since $tested \subseteq enrolled$.

    \begin{schema}{AlreadyEnrolled}
    \Xi Class \\
    S? : Student \\
    R! : EnrollResponse \\
    \where
    S? \in enrolled \\
    S? = alreadyenrolled \\
    \end{schema}

Therefore, the enroll operation must be the disjunction of EnrollOK and
AlreadyEnrolled.

    \begin{zed}
    Enroll \triangleq EnrollOK \vee AlreadyEnrolled
    \end{zed}

\section*{Operation - Test}

When an enrolled student is tested, the student passes and is now part of the
student population in the class marked as tested.

    \begin{schema}{TestOK}
    \Delta Class \\
    S? : Student \\
    R! : TestResponse \\
    \where
    S? \in enrolled \\
    S? \notin tested \\
    tested' = tested \union \left\{ S? \right\} \\
    enrolled' = enrolled \\
    R! = success \\
    \end{schema}

If a student has not enrolled, then the attempt to test the student must
necessarily fail.

    \begin{schema}{NotEnrolled}
    \Xi Class \\
    S? : Student \\
    R! : TestResponse \\
    \where
    S? \notin enrolled \\
    S? = notenrolled \\
    \end{schema}

In addition, a student can not be tested twice.

    \begin{schema}{AlreadyTested}
    \Xi Class \\
    S? : Student \\
    R! : TestResponse \\
    \where
    S? = tested \\
    R! = alreadytested \\
    \end{schema}

Therefore, the test operation must be the disjunction of TestOK, NotEnrolled,
and AlreadyTested.

    \begin{zed}
    Test \triangleq TestOK \vee NotEnrolled \vee AlreadyTested
    \end{zed}

\end{document}
